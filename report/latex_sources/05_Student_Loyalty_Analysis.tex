\documentclass[12pt,a4paper]{article}
\usepackage[utf8]{inputenc}
\usepackage{graphicx}
\usepackage{geometry}
\usepackage{enumitem}
\usepackage{titlesec}
\usepackage{hyperref}

\geometry{margin=2.5cm}

\title{Student Loyalty Analysis}
\date{}

\begin{document}

\maketitle

\section*{1. Loyalty Distribution}
As shown in Figure~\ref{fig:loyalty-distribution}, the majority of students (244 individuals) registered for only one course, highlighting that most students participate only once. In contrast, only 35 students enrolled in two courses and just 13 students participated in three courses. This means over 83\% of students are non-loyal (single-course attendees).

This suggests that while the academy attracts a large number of students, retention and repeat engagement are limited, and there may be untapped potential in encouraging re-enrolment.

\begin{figure}[h!]
    \centering
    \includegraphics[width=0.9\textwidth]{Student Loyalty Distribution (Number of Courses Attended).png}
    \caption{Student Loyalty Distribution (Number of Courses Attended)}
    \label{fig:loyalty-distribution}
\end{figure}

\section*{2. Income Contribution by Loyalty Segment}
Figure~\ref{fig:loyalty-income} shows that despite their smaller number, loyal students (those who registered for more than one course) contributed approximately 25.7\% of total income, while non-loyal students were responsible for 74.3\%.

Although loyal students are a minority, their higher lifetime value is evident. This finding underscores the importance of developing targeted strategies (e.g., loyalty programs, follow-up offers) to increase repeat registrations.

\begin{figure}[h!]
    \centering
    \includegraphics[width=0.8\textwidth]{Income from Loyal and Non-loyal Students (Pie Chart).png}
    \caption{Income from Loyal and Non-loyal Students}
    \label{fig:loyalty-income}
\end{figure}

\section*{Summary}
\begin{itemize}
    \item Most students register once and do not return.
    \item Loyal students generate disproportionately higher income.
    \item Improving student retention could significantly enhance revenue without requiring a proportional increase in new student acquisition.
\end{itemize}

\section*{Recommendation}
Introduce loyalty programs, discounts for repeat enrolments, or targeted follow-ups to increase student retention and maximise revenue.

\end{document}