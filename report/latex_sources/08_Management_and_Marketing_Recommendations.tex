\documentclass[12pt,a4paper]{article}
\usepackage[utf8]{inputenc}
\usepackage{graphicx}
\usepackage{geometry}
\usepackage{enumitem}
\usepackage{titlesec}
\usepackage{hyperref}

\geometry{margin=2.5cm}

\title{Management and Marketing Recommendations}
\date{}

\begin{document}

\maketitle

%\section*{Management and Marketing Recommendations}


\section*{1.  Reach More Students in Other Areas}
\textit{Finding:} Most students and most of the money come from Shiraz and Fars province. Other areas only bring in a small number of students. \\
\textit{Recommendation:} Focus advertising in provinces where we don’t have many students yet (like Hormozgan, Bushehr, or Kerman) to reach more people and bring in income from new places.

\section*{2. Encourage Students to Come Back for More Courses}
\textit{Finding:} More than 80\% of students attend only one course. But students who take more than one course bring in a lot more income. \\
\textit{Recommendation:} Create loyalty offers like discounts, point systems, or membership cards. Also, stay in touch with students after their first course to invite them back.

\section*{3. Offer Courses as Packages or in a Recommended Order}
\textit{Finding:} Many students sign up for more than one course at the same time. Popular combinations are burger + pizza, or sausage + nugget. \\
\textit{Recommendation:} Offer two or more courses together as a package with a better price. Also, suggest the next course to take based on what students usually choose together.

\section*{4. Plan Courses Based on the Season}
\textit{Finding:} Summer and autumn are the busiest times with more students and more income. But course prices are not always adjusted to match the higher demand. \\
\textit{Recommendation:} Schedule high-demand courses in Summer/Autumn, and use Spring/Winter for promotional campaigns or discounted classes to balance demand.

\section*{5. Adjust Courses for Different Regions}
\textit{Finding:} Some courses (e.g., kabab, traditional) are only popular in specific regions/seasons. \\
\textit{Recommendation:} Customise course offerings by region and season to match local preferences; avoid a one-size-fits-all approach.

\section*{6. Marketing Insights}
\textit{Finding:} Most classes are held close to the weekend. Also, students often sign up for more than one course at the same time. \\
\textit{Recommendation:} Run ads and send messages earlier in the week to match the upcoming class days. You can also encourage students to register for more than one course by offering combo deals.

\section*{7. Implement Feedback and Follow-up Systems}
\textit{Finding:} Lack of insight into post-course satisfaction and reasons for not returning. \\
\textit{Recommendation:} Send short surveys after the course and follow up with emails or messages. This can help improve the courses and encourage students to return.

\section*{8. Try New Offers in Less Active Regions}
\textit{Finding:} Some new regions are slowly growing, even if they’re still small. \\
\textit{Recommendation:} Test short-term offers or beginner discounts in a few selected provinces. See what works, and then expand the successful ideas to other areas.

\end{document}