\documentclass[12pt,a4paper]{article}
\usepackage[utf8]{inputenc}
\usepackage{graphicx}
\usepackage{geometry}
\usepackage{enumitem}
\usepackage{titlesec}
\usepackage{hyperref}

\geometry{margin=2.5cm}

\title{Management and Marketing Recommendations}
\date{}

\begin{document}

\maketitle

%\section*{Management and Marketing Recommendations}


\section*{1.  Reach More Students in Other Areas}
\textbf{Finding:} Most students and most of the income come from Shiraz and Fars province. Other areas contribute only a small number of students. \\
\textbf{Recommendation:} Focus advertising in provinces with low student numbers (such as Hormozgan, Bushehr, or Kerman) to reach new audiences and generate additional income.

\section*{2. Encourage Students to Come Back for More Courses}
\textbf{Finding:} More than 80\% of students attend only one course. But students who take more than one course bring in a lot more income. \\
\textbf{Recommendation:} Create loyalty offers like discounts, point systems, or membership cards. Also, stay in touch with students after their first course to invite them back.

\section*{3. Offer Courses as Packages or in a Recommended Order}
\textbf{Finding:} Many students sign up for more than one course at the same time. Popular combinations are burger + pizza, or sausage + nugget. \\
\textbf{Recommendation:} Introduce bundled course packages at preferential prices. Also, suggest the next course to take based on what students usually choose together.

\section*{4. Plan Courses Based on the Season}
\textbf{Finding:} Summer and autumn are the busiest times with more students and more income. But course prices are not always adjusted to match the higher demand. \\
\textbf{Recommendation:} Schedule high-demand courses in Summer/Autumn, and use Spring/Winter for promotional campaigns or discounted classes to balance demand.

\section*{5. Adjust Courses for Different Regions}
\textbf{Finding:} Some courses (e.g., kabab, traditional) are only popular in specific regions/seasons. \\
\textbf{Recommendation:} Customise course offerings by region and season to match local preferences; adopt a tailored approach instead of a one-size-fits-all model.

\section*{6. Refine Marketing Timing}
\textbf{Finding:} Most classes are held close to the weekend, and students often sign up for multiple courses at the same time. \\
\textbf{Recommendation:} Consider running ads and sending messages earlier in the week to align with upcoming class days. Students can also be encouraged to register for multiple courses through combo deals.

\section*{7. Implement Feedback and Follow-up Systems}
\textbf{Finding:} Lack of insight into post-course satisfaction and reasons for not returning. \\
\textbf{Recommendation:} Send short surveys after the course and follow up with emails or messages. This can help improve the courses and encourage students to return.

\section*{8. Try New Offers in Less Active Regions}
\textbf{Finding:} Some new regions are slowly growing, even if they’re still small. \\
\textbf{Recommendation:} Test short-term offers or beginner discounts in a few selected provinces. Evaluate the effectiveness of these offers and expand successful initiatives to other regions.

\end{document}